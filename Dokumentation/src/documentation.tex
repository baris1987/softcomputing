\documentclass[10pt,a4paper]{article}
\usepackage[ngerman]{babel}
\usepackage[utf8]{inputenc}
\usepackage{graphicx}
\usepackage{titling}
\bibliographystyle{alpha}
\usepackage[left=2.5cm,right=2.5cm]{geometry}
\usepackage{floatrow}

%PDF Hyperlinks
\usepackage{hyperref}
\hypersetup{pdftex}
\usepackage{hypcap}

% Add subtitle-command
\newcommand{\subtitle}[1]{%
  \posttitle{%
    \par\end{center}
    \begin{center}\large#1\end{center}
    \vskip0.5em}%
}

\begin{document}
\begin{titlepage}
\vspace*{1cm}
\begin{center}
\Huge
Technische Hochschule Nürnberg\\
\vspace*{2cm}
\large
 {\Large Projektarbeit im Wahlpflichtfach Softcomputig\\
Dozent: Prof. Dr. Reinhard Eck\\}
\vspace*{2cm}
\Huge
Neuronale Netze\\
\large
\vspace*{1cm}
Praktische Anwendung bei der automatischen Erkennung von Captchas
\vspace{1cm}

\vspace{2cm}

 \begin{tabular}{p{6 cm}p{6 cm}}
    	vorgelegt von & {Christopher Althaus} \\
		& {Baris Akdag} \\
		& {Matthias Jentsch} \\
		\\ & \\
    	Abgabe:& 1. Juli 2013
 \end{tabular}\\
    


\end{center}
\end{titlepage}

\tableofcontents

% Hier bitte alle selbst geschriebenen Kapiel einfügen. 
\include{Fachartikel}
% Motivation, Einsatz im täglichem Leben...,  //Baris
% Captcha Generator //Baris
% Lösungsansatz, Grober Programm Ablauf, Aufbau des Neuronalen Netzes //Matze
% Bild aufbearbeitung (Toolbox image)// Matze
% Segmentierung Klassifizierung // Christopher

% Anleitung zur Benutzung // Christopher

% Fazit // Baris + Überarbeitung 
\include{Anhang}
%Codedokumentation
\nocite{*}
\bibliography{Quellen.bib}

\end{document}
