\documentclass[10pt,a4paper]{article}
\usepackage[ngerman]{babel}
\usepackage[utf8]{inputenc}
\usepackage{graphicx}
\usepackage{titling}
\bibliographystyle{alpha}
\usepackage[left=2.5cm,right=2.5cm]{geometry}
\usepackage{floatrow}
\usepackage{wrapfig}


%PDF Hyperlinks
\usepackage{hyperref}
\hypersetup{pdftex}
\usepackage{hypcap}

\usepackage{color}
\definecolor{green}{rgb}{0.0, 0.65, 0.31}
\usepackage{listings} 
\lstset{
	numbers=left, 
	numbersep=12pt,
	commentstyle=\color{green},
	basicstyle=\small
} 
\usepackage{hyperref}
\lstset{language=Matlab} 


% Add subtitle-command
\newcommand{\subtitle}[1]{%
  \posttitle{%
    \par\end{center}
    \begin{center}\large#1\end{center}
    \vskip0.5em}%
}

\begin{document}
\begin{titlepage}
\vspace*{1cm}
\begin{center}
\Huge
Technische Hochschule Nürnberg\\
\vspace*{2cm}
\large
 {\Large Projektarbeit im Wahlpflichtfach Softcomputig\\
Dozent: Prof. Dr. Reinhard Eck\\}
\vspace*{2cm}
\Huge
Neuronale Netze\\
\large
\vspace*{1cm}
Praktische Anwendung bei der automatischen Erkennung von Captchas
\vspace{1cm}

\vspace{2cm}

 \begin{tabular}{p{6 cm}p{6 cm}}
    	vorgelegt von & {Christopher Althaus} \\
		& {Baris Akdag} \\
		& {Matthias Jentsch} \\
		\\ & \\
    	Abgabe:& 1. Juli 2013
 \end{tabular}\\
    


\end{center}
\end{titlepage}

\tableofcontents

% Hier bitte alle selbst geschriebenen Kapiel einfügen. 
%\include{Fachartikel}
% Motivation, Einsatz im täglichem Leben...,  						//Baris
% Captcha Generator 
\newpage
\section{Grundlagen}

\subsection{Was sind Captchas?}
Die Bezeichnung Captcha steht für "Completely Automated Public Turing test to tell Computers and Humans Apart". Mit Hilfe von Captchas soll also ein Mensch von einem Computer unterschieden werden können. Der Computer könnte  beliebig oft Passwörter aufprobieren, bis er das korrekte Passwort gefunden hat. Wenn jedoch ab dem dritten Versuch eine Validierung des Menschen in Form eines Captchas abgefragt wird, wird dieses Vorgehen deutlich erschwert. Das Prinzip des Captchas basiert auf einer Aufgabe (Challenge) und einer richtigen Antwort (Response). Die Aufgabe sollte für einen Menschen möglichst sehr einfach für einen Computer jedoch sehr schwer lösbar sein. Die Aufgabe stellt sich sehr oft in Form eines Bildes dar. Auf diesem Bild kann bspw. eine schwer lesbare Zeichenfolge von Buchstaben und Ziffern oder auch ein Objekt, wie bspw. ein Hund abgebildet sein. Der Mensch erkennt einen Hund auf einem Bild sofort, jedoch ist die softwareseitige Erkennung  schwer implementierbar. Ein Mensch erkennt einen Hund auch, wenn er die Hunderasse vorher noch nie gesehen hat, weil er in seinem neuronalen Netz im Gehirn lernt, kombiniert und erkennt. Viele Captchas sind Zeichenfolgen, so dass der Mensch die auf dem Bild zu sehende Zeichenfolge in ein Textfeld eingeben muss. Die Zeichenfolge kann dabei etwas gedreht oder gedehnt sein, damit diese nicht einfach durch eine Texterkennungssoftware gelesen werden kann. Captchas werden auch HIP (Human Interaction Proof) genannt.

\begin{figure}[htbp]
  \centering
  \fbox{
    \includegraphics[scale=0.5]{res/captchas.png}
  }
  \caption{Verschiedene Zeichenfolgen Captchas aus der Praxis.}
  \label{Captchas}
\end{figure}

\begin{figure}[htbp]
  \centering
  \fbox{
    \includegraphics[scale=0.5]{res/catpcha-example.png}
  }
  \caption{Captcha Anwendung bei Facebook.}
  \label{Captchas}
\end{figure}

\subsection{Motivation}
Ziel dieses Softcomputing Projektes ist es, reale Captchas mit Hilfe von neuronalen Netzen zu dekodieren. Um die Entwicklung zu vereinfachen, wurde Matlab verwendet, da darin bereits Funktionen zur Bildverarbeitung und neuronalen Netzen integriert sind. Dabei wird die Verwendung von neuronalen Netzen studiert und angewandt. Die verwendeten Captchas sind Zeichenfolgen von Buchstaben und Grafiken, die leicht gedreht und mit Mustern im Hintergrund schwerer lesbar gemacht wurden. Es soll gezeigt werden, dass mit einer größeren Menge Captchas das neuronale Netz aufgebaut bzw. trainiert werden kann, um anschließend ähnliche Captchas zu dekodieren. Damit kann die Unterscheidung eines Menschen von einem Computer aufgehoben werden, um bspw. Softwaresysteme angreifbar zu machen oder deren Captchas zu verbessern. 

\subsection{Captcha Generator}
Für den Aufbau des neuronalen Netzes werden viele Captchas inklusive der Lösung benötigt. Um an diese Menge von Captchas zu kommen wurde ein eigener Captcha Generator entwickelt. Der Generator wurde als eine .NET Applikation in C\# realisiert. Der Captcha Generator hat einige Konfigurationsmöglichkeiten: Ausgabeverzeichnis, Anzahl der Zeichen auf einem Bild, Anzahl der Captchas und der Neigungswinkel der einzelnen Zeichen in Grad. Der Generator erzeugt die Captachs im PNG-Bildformat und setzt als Dateinamen die Lösung der Zeichenfolge. Die Captchas können auf diese Art und Weise genutzt werden, um das neuronale Netz aufzubauen und die Erkennung mit dem Dateinamen des Bildes zu validieren.

\begin{figure}[htbp]
  \centering
  \fbox{
    \includegraphics{res/captcha-generator.png}
  }
  \caption{Benutzeroberfläche des Captcha Generators.}
  \label{Captchas}
\end{figure}

\begin{figure}[htbp]
  \centering
  \fbox{
    \includegraphics{res/generator-captchas.png}
  }
  \caption{Drei mit dem Generator erzeugte Captchas.}
  \label{Captchas}
\end{figure}


% Lösungsansatz, Grober Programm Ablauf, Aufbau des Neuronalen Netzes //Matze
\section{Konzept}

Diese Sektion enthält eine Beschreibung des generellen Lösungsansatzes, der in
den einzelnen Phasen des Programmes verfolgt wird. Eine detaillierte Dokumentation
der konkreten Umsetzung befindet sich in Sektion \ref{module} - Modulbeschreibung.

Um eine hohe Erkennungsrate der Captchas zu erreichen und den Rechenaufwand
gering zu halten, haben wir uns dazu entschieden
uns nicht ausschließlich auf die Funktion des neuronalen Netzes zu verlassen
und stattdessen im Programmablauf Vorbereitungsphasen zu integrieren, die die
Daten für das neuronale Netz aufbereiten. Durch diese Aufbereitung wollen wir
die Zahl der Inputneuronen gering halten und eventuelle Störsignale bereits Image
Voraus entfernen.

Aufgrund dieser Anforderungen haben wir uns bei der Umsetzung für Matlab entschieden, 
da es bereits passende Erweiterungen zur Aufbereitung von Bildern und zum
Erstellen von neuronalen Netzen speziell für die Mustererkennung bietet.


\subsection{Ablaufphasen des Programms}

Der Ablauf des Programms kann in folgende grobe Phasen unterteilt werden: 

\begin{itemize}

\item Trainingsphase

Erstellen des neuronalen Netzes.


\item Einlesen und Aufbereiten der Captchas

Liest die Captchas aus den Dateien ein und entfernt grobe Störsignale.


\item Segmentierung

Teilt die aufbereiteten Bilder in einzelne Zeichen und normalisiert die
Ausrichtung.


\item Datenaggregation

Zusammenfassen von verwandten Datensätzen um die Anzahl der Eingaben zu reduzieren.


\item Klassifizierung

Erkennen der Zeichen durch das neuronale Netz

\end{itemize}


\subsubsection{Trainingsphase}

Die Trainingsphase ist die Phase in der ein neuronales Netz zum Erkennen der
Zeichen erstellt wird. Die Trainingsphase soll nur dann aufgerufen werden, wenn
nicht bereits ein Vorbereitetes Netz existiert. In der Trainingsphase werden
vorbereitete Captchas zusammen mit der Lösung eingelesen und analog zu den
normalen Bildern aufbereitet. Die einzelnen Zeichen werden zusammen mit dem
Ergebnis zum Trainieren des Neuronalen Netzes verwendet. Die Trainingsphase
soll so lange andauern, bis sich zwischen den einzelnen Trainingsphasen keine
relevante Verbesserung ergibt.

\subsubsection{Einlesen und Aufbereiten der Captchas}

\begin{wrapfigure}[6]{r}{5cm}
  \begin{center}
  \vspace{-48pt}
    \includegraphics[width=4cm]{res/Aufbereitung.png}
  \end{center}
  \caption{Aufbereitung eines Captchas}
\end{wrapfigure}

In diesem Schritt werden alle Bilder aus einem Eingabeverzeichnis eingelesen und
aufbereitet. Die Aufbereitung versucht alle Pixel, die nicht zum eigentlichen
Text gehören herauszufiltern. Zudem werden Farben durch binäre Werte ersetzt
die entweder Schwarz oder Weiß darstellen. Hierbei wird anhand eines
Schwellwertes unterschieden ab wann ein Graustufenwert als Schwarz oder als
Weiß gilt. Eine genauere Beschreibung der Aufarbeitung der Bilder befindet sich
in Kapitel \ref{images} - Aufbereitung von Bildern mit der Matlab-Image-Toolbox.


\subsubsection{Segmentierung}

  \begin{wrapfigure}[6]{R}[0pt]{6cm}
  \vspace{-35pt}
  \begin{center}
    \includegraphics[width=5cm]{res/Segmentierung.png}
  \end{center}
  \vspace{-5pt}
  \caption{Segmentierung}
  \vspace{-10pt}
\end{wrapfigure}


Bei der Segmentierung wird versucht die Zeichen des Worts in einzelne Segmente
zu unterteilen. Da das neuronale Netz eine konstante Menge an Eingabewerten
benötigt, wird zudem die Auflösung der Segmente auf eine Einheitliche größe
Festgelegt. Alle Zeichen werden auf die gleiche Art ausgerichtet um
Verschiebungen auszugleichen.

\subsubsection{Datenaggregation}

\begin{wrapfigure}[15]{R}[0pt]{7cm}
  \begin{center}
    \includegraphics[width=6cm]{res/Aggregation.png}
  \end{center}
  \vspace{-5pt}
  \caption{Aggregation der Spalten/Zeilen}
  \vspace{-10pt}
\end{wrapfigure}


Bei der Datenaggregation wird versucht die Menge an Eingabewerten zu reduzieren,
ohne dabei viel Genauigkeit zu verlieren. Dies wird erreicht indem 
die Summen aller Schwarzen Pixel pro Zeile und Spalte gebildet wird. 

Ohne diese Aggregation, also bei direkter Eingabe aller Pixel in das
neuronale Netz, wäre die Laufzeit des Lernvorgangs und der Mustererkennung zu
hoch um das Programm sinnvoll verwenden zu können.
\subsubsection{Klassifizierung}

Die Klassifizierung benutzt das neuronale Netz, um die eigentliche
Mustererkennung durchzuführen und so die einzelnen Buchstaben zu
erkennen. Jede Summe aus der Datenaggregation wird als einzelner Eingabewert
für das Netz verwendet. Diese Werte sollen in Matlab in Form eines Vektors zu
einem Set aus Eingabewerten für ein einzelnes Zeichen zusammengefasst werden.

\section{Struktur des neuronalen Netzes}

\begin{wrapfigure}[15]{R}[0pt]{7cm}
  \begin{center}
    \includegraphics[width=12cm]{res/PatternNet-Aufbau.png}
  \end{center}
  \vspace{-5pt}
  \caption{Aufbau des Neuronalen Netzes}
  \vspace{-10pt}


\end{wrapfigure}
Um festzustellen wie viele Hidden-Neuronen das Netz braucht, wurde die Leistungsfähigkeit des Netzes bei verschiedensten Mengen an Neuronen getestet. Da der Anstieg der Leistungsfähigkeit ab $38$ Hidden-Neuronen aufhört, legen wir die Anzahl dieser auf $38$ fest.

\subsection{Matlab - Die Klasse ``Patternnet''}

In der Matlab-Toolbox für neuronale Netze existiert bereits ein vorgefertigtes
neuronales Netz für die Mustererkennung. Diese Klasse nennt sich ``Patternnet''
und erbt von der Basisklasse ``nnet''. Die Klasse erstellt ein Feedforward-Netz
mit der Lernfunktion ``trainscg'', die die Skalierte Gradientenmethode zum
Berechnen der Extrema verwendet.


\subsection{Transferfunktion}
 Als Transferfunktion wird die Funktion``tansig'' verwendet, die eine Gleichung der folgenden Form darstellt:

\begin{equation}
n = tansig(n) =  \frac{2}{(1 + e^{-2*n})} - 1
\end{equation}

\subsection{Skalierte Gradientenmethode}

Die Skalierte Gradientenmethode wird verwendet um die Gewichtungen der
Transferfunktion zu aktualisieren. 

\newpage
\section{Aufbereitung von Bildern mit der Matlab-Image-Toolbox}
\label{images}


% Bild aufbearbeitung (Toolbox image)								// Matze
\include{bildaufbereitung}
% Segmentierung Klassifizierung 									// Christopher

% Anleitung zur Benutzung 											// Christopher
\newpage
\section{Bedienungsanleitung}
\subsection{Voraussetzungen}
Um zu gewährleisten, dass die Matlab Module dieses Projekts ordnungsgemäß funktionieren, wird eine Matlab Version R2013a oder höher mit folgenden Toolboxen benötigt:
\begin{itemize}
\item Neural Network Toolbox
\item Image Processing Toolbox
\end{itemize}
Für die Ausführung des Captcha Generators wird lediglich das .Net Framework 2.0 benötigt.

\subsection{Ordnerstruktur}
Da die Matlab Module auf eine fest definierte Ordnerstruktur zugreifen, ist es wichtig, diese einzuhalten. Die Ordnerstruktur sollte der folgenden entsprechen.
\begin{verbatim}
Matlab/
  images/
    Test/
      Capture1.png
      Capture2.png
      ...
    Training/
      0ZEHSQ.png
      TWU7P8.png
      ...
  src/
    buildNetwork.m
    classify.m
    preprocess.m
    recognize.m
    runTest.m
    segment.m
\end{verbatim}
Wichtig ist, dass die Ordner, in welchen sich die Captchas befinden, einmal zum Matlab Pfad hinzugefügt wurden. Dies kann mit Hilfe des Matlab Datei Explorers realisiert werden:
\begin{verbatim}
Rechtsklick auf den Ordner 'images'
  -> Add to Path
    -> Selected Folders and Subfolders
\end{verbatim}
Wenn die Ordner zum Pfad hinzugefügt wurden, sollten diese grau hinterlegt dargestellt werden. Weiterhin ist zu beachten, dass man sich während der Ausführung der Module stets im Ordner ''/src'' befinden muss. Dies ist erst der Fall, wenn in der Pfadangabe des aktuellen Ordners von Matlab '/src' steht!
\subsection{Generator}
Sollen neue Captchas generiert werden, kann beim Generator unter Target der Zielordner angegeben werden, in welchem die Captchas gespeichert werden sollen. Unter Files wird die Anzahl der zu erzeugenen Captchas festgelegt und Chars legt die Anzahl der Zeichen pro Captcha fest. Es gilt zu beachten, dass die Matlab Module auf sechs Zeichen pro Captcha programmiert wurden. Deshalb kommt es ohne entsprechende Codeanpassungen zu Fehlern, wenn mehr oder weniger Zeichen in einem Captcha enthalten sind.\\
Mittels Angle lässt sich der Wertebereich des Neigungswinkels der Zeichen im Captcha eingrenzen.\\
Die Generierung der Captchas wird mittels des Start Button gestartet. Sobald die Captchas generiert wurden, erscheint eine Messagebox, die die erfolgreiche Erstellung bestätigt.

\subsection{Captcha Erkennung}
Die Erkennung von Captchas erfolgt ausschließlich in Matlab. Für eine erfolgreiche Erkennung ist es wichtig, dass der Ordner, in welchem sich das Bild befindet, zum Matlab Pfad hinzugefügt wurde. Wie dies funktioniert, wird unter dem Punkt Ordnerstruktur erklärt.\\
Vor der Ausführung muss sich Matlab im Verzeichnis ''/src'' der Ordnerstruktur befinden. Die Abfrage eines einzelnen Captchas erfolgt mittels des \textit{recognize} Moduls, welches als Parameter den Namen der Bilddatei erwartet. Der Abruf erfolgt durch das Kommandofenster innerhalb von Matlab:
\begin{verbatim}
>> recognize('ZVEEIO.png')

ans =

ZVEEIO
\end{verbatim} 
Wird die Anfrage nicht mit einem Semikolon abgeschlossen, gibt Matlab das Ergebnis direkt im Kommandofenster aus. Alternativ ist es auch möglich, das Ergebnis der Abfrage einer Variable zuzuweisen:
\begin{verbatim}
>> capture = recognize('capture.png');
>> capture

capture =

ZVEEIO
\end{verbatim} 
$\;$ \\
Sollen mehrere Captchas gelöst werden, so können diese in den Ordner ''/Test'' der Ordnerstruktur gespeichert werden. Durch den Aufruf des Moduls \textit{runTest}, werden automatisch alle Captchas durchlaufen und die Ergebnisse direkt im Kommandofenster ausgegeben:
\begin{verbatim}
>> runTest();
1	 Aktuell: 0E0HSQ	 Erkannt: 0E0HSQ	 OK
2	 Aktuell: 0JDDQ2	 Erkannt: 0JDDQ2	 OK
3	 Aktuell: 0K42MH	 Erkannt: 0K42MH	 OK
4	 Aktuell: 1VVYE7	 Erkannt: 1VVYE7	 OK
5	 Aktuell: 1XHLLA	 Erkannt: 1XHLLA	 OK
\end{verbatim} 

%Codedokumentation													//Christopher
\newpage
\section{Modulbeschreibung}
Zur Umsetzung des Projekts wurde verschiedene Module implementiert, welche im Folgenden der Reihe nach, näher beschrieben werden sollen.
\subsection{Preprocessor}
Da die Captcha vom Captcha Generator mit Artefakten versehen worden sind, welche das maschinelle Erkennen erschweren sollen, wird im Modul ''preprocess'' versucht diese Artefakte vollständig zu entfernen, welches im folgendem dargestellt ist.
\begin{lstlisting}
function cleaned = preprocess(image)

	greyScale = rgb2gray(image);
	cleaned = greyScale < 65;

end
\end{lstlisting}
Das Modul nimmt als einzigen Parameter ein Capture als Bild entgegen und gibt als Rückgabewert das von Artefakten bereinigte Bild zurück. Da in diesem Projekt die Neuronalen Netze im Vordergrund standen, ist die Bereinigung des Bildes denkbar einfach. Die vom Generator erzeugten Artefakte haben stets einen Grauwert, welcher über 65 liegt. Deshalb wird zunächst die Matlab Funktion 'rgb2gray' auf das übergebene Bild angewendet, welche das Bild in Graustufen umwandelt.\\
Anschließend werden in der Matrix cleaned, die Bildpunkte, deren Grauwert kleiner 65 ist auf eins gesetzt und alle anderen auf null.



% Fazit 															// Baris
\newpage
\section{Fazit}
Ziel des Projekts war es, Captchas mit Hilfe von neuronalen Netzen zu dekodieren. Das Projekt ermöglichte uns eine tiefgehende Einarbeitung in neuronale Netze. Des Weiteren konnten wir erste Erfahrungen in Matlab und der Bildverarbeitung machen. Durch die einzelnen Teilschritte Preprocessing, Segmentierung und Klassifizierung konnten wir das Problem in kleinere Arbeitspakete einordnen. Durch die Bildverarbeitungsbibliothek und der Neuronal Network Toolbox in Matlab konnten wir auf stabile und gut dokumentierte Funktionen zugreifen. Beim Aufbau des neuronalen Netzes mussten wir auf verschiedene Aspekte, wie die Anzahl der Hidden Neuronen, die Trainingsphase und die Erkennungsrate, achten. [...] Die Erkennungsrate unseres neuronalen Netzes liegt meist bei über 90\%.

\nocite{*}
\bibliography{Quellen.bib}

\end{document}
