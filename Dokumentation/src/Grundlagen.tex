\section{Grundlagen}

\subsection{Was sind Captchas?}
Die Bezeichnung Captcha steht für "Completely Automated Public Turing test to tell Computers and Humans Apart". Mit Hilfe von Captchas soll ein Menschen von einem Computer unterschieden werden können. Das ist notwendig, um bspw. bei einem Login in ein Softwaresystem den Menschen von einem Computer zu unterscheiden. Der Computer könnte sonst beliebig oft Passwörter aufprobieren bis es das korrekte Passwort gefunden hat. Wenn jedoch ab dem dritten Versuch eine Validierung des Menschen in Form eines Captchas abgefragt wird, wird dieses Vorgehen deutlich erschwert. Das Captcha Prinzip ist eine Aufgabe (Challenge) und eine richtige Antwort (Response). Die Aufgabe sollte für einen Menschen möglichst sehr einfach lösbar aber für einen Computer sehr schwer sein. Die Aufgabe besteht sehr oft in Form eines Bildes. Auf diesem Bild kann bspw. eine schwer lesbare Zeichenfolge von Buchstaben und Ziffern sein, aber auch ein Objekt, wie bspw. ein Hund. Der Mensch erkennt einen Hund auf einem Bild sofort, jedoch ist diese Erkennung softwareseitig schwer implementierbar. Ein Mensch erkennt einen Hund auch, wenn er die Hunderasse vorher noch nie gesehen hat, weil er in seinem neuronalen Netz im Gehirn lernt, kombiniert und erkennt. Die meisten Captchas sind jedoch Zeichenfolgen, so dass der Mensch die auf dem Bild zu sehende Zeichenfolge in ein Textfeld eingeben muss. Die Zeichenfolge kann dabei etwas gedreht oder gedehnt sein, damit diese nicht einfach durch eine Texterkennungssoftware gelesen werden kann. Captchas werden auch HIP (Human Interaction Proof) genannt.

\subsection{Motivation}
Ziel dieses Softcomputing Projekts ist es, reale Captchas mit Hilfe von neuronalen Netzen in MATLAB dekodieren zu können. Dabei soll die Verwendung von neuronalen Netzen studiert und angewandt werden. Als Captchas dienen Zeichenfolgen von Buchstaben und Grafiken, die leicht gedreht und mit Mustern im Hintergrund schwerer lesbar gemacht wurden. Es soll gezeigt werden, dass mit einer Menge von 1000 Captchas das neuronale Netz aufgebaut bzw. trainiert wird, um anschließend ähnliche Captchas dekodieren zu können. Damit kann die Unterscheidung eines Menschen von einem Computer gebrochen werden, um bspw. Softwaresysteme angreifbar zu machen.

\subsection{Captcha Generator}
Für den Aufbau des neuronalen Netzes werden viele Captchas inklusive der Lösung benötigt. Um an diese Menge von Captchas zu kommen wurde ein eigener Captcha Generator entwickelt. Der Generator wurde in einer einfachen .NET Applikation in C\# realisiert. Der Captcha Generator hat einige Konfigurationsmöglichkeiten: Ausgabeverzeichnis, Anzahl der Zeichen auf einem Bild, Anzahl der Captchas und der Neigungswinkel der Zeichen in Grad. Der Generator erzeugt nun die Captachs im PNG-Bildformat und setzt als Dateinamen die Lösung der Zeichenfolge auf dem Bild. Die Captchas können auf diese Art und Weise genutzt werden, um das neuronale Netz aufzubauen und die Erkennung mit dem Dateinamen des Bildes zu validieren.