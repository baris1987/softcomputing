\section{Fazit}
Ziel des Projekts war es, Captchas mit Hilfe von neuronalen Netzen zu dekodieren. Dies ermöglichte uns eine tiefgehende Einarbeitung in neuronale Netze. Des weiteren konnten wir erste Erfahrungen in Matlab und der Bildverarbeitung machen. Durch die einzelnen Teilschritte Preprocessing, Segmentierung und Klassifizierung konnten wir das Problem in kleinere Arbeitspakete einordnen und so auch zusätzliche Erfahrungen im Bereich der modularen Softwareentwicklung sammeln. Durch die Bildverarbeitungsbibliothek und der Neuronal Network Toolbox in Matlab konnten wir auf stabile und gut dokumentierte Funktionen zugreifen. Beim Aufbau des neuronalen Netzes mussten wir auf verschiedene Aspekte, wie die Anzahl der Hidden Neuronen, die Trainingsphase und die Erkennungsrate, achten.\\
Die Entscheidung ein neuronales Netz für die Mustererkennung zu verwenden, hat sich als richtig erwiesen, da die Erkennungsrate meist bei um die 80\% bis 90 \% liegt. Für eine praktische Anwendung würde jedoch auch eine deutlich geringere Erkennungsrate ausreichen, da bei Erkennungstests durch Captchas meist sehr viele Versuche möglich sind. In der Realität würde sich der Schritt des Preprocessings wesentlich aufwendiger gestalten, da die Captchas bewusst so entwickelt werden dass eine automatische Erkennung nur schwer möglich ist. Die erschwerte Erkennung resultiert durch Artefakte, Drehung und Deformierung der Zeichen im Captcha, wodurch eine automatische Segmentierung der Zeichen nicht ohne weiteres möglich ist.
